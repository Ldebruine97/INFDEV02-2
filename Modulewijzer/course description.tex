\subsection{Theoretical examination DEV II}
The general shape of a theoretical exam for \texttt{DEV II} is made up of a series of highly structured open questions.


\paragraph{Question I: abstracting patterns with functions} \ \\

\textbf{General shape of the question:} \textit{Given the following block of code, define functions in order to reduce repetition, and rewrite a shorter but equivalent version of the original code by using your functions.}

\ \\ 

\textbf{Concrete example of question:} \textit{Given the following block of code, define a function in order to reduce repetition, and rewrite a shorter but equivalent version of the original code by using your functions.}

\begin{lstlisting}
x = 10
y = x + 1
z = x + y * 2
print(x,y,z)

a = 5
b = a + 2
c = a + b * 4
print(a,b,c)
\end{lstlisting}

\ \\ 

\textbf{Concrete example of answer:} \textit{The resulting code is:}

\begin{lstlisting}
f(s, d, f) =
  v0 = s
  v1 = s + d
  v2 = v0 + v1 * f
  print(v0,v1,v2)

f(10,1,2) 
f(5,2,4) 
\end{lstlisting}

\ \\ 

\textbf{Points:} \textit{25\%.}

\ \\ 

\textbf{Grading:} \textit{All points for correct answer, correct function but wrong use or correct use but function with minor mistakes half points, wrong function and wrong use zero points.}

\ \\ 

\textbf{Associated learning goals:} \texttt{FUNABS}, \texttt{FUNDEF}.

\ \\ 

\paragraph{Question II: abstracting patterns and structures with classes} \ \\

\textbf{General shape of the question:} \textit{Given the following block of code, define a series of classes in order to reduce repetition of variables and code, and rewrite a shorter but equivalent version of the original code by using your classes.}

\ \\ 

\textbf{Concrete example of question:} \textit{Given the following block of code, define a class in order to reduce repetition of variables and code, and rewrite a shorter but equivalent version of the original code by using your class.}

\begin{lstlisting}
cnt1 = 0
incr1() = cnt1 <- cnt1 + 1

cnt2 = 1
incr2() = cnt2 <- cnt2 * 2

for i = 1 to 100 do
  incr1()
  incr2()
\end{lstlisting}

\ \\ 

\textbf{Concrete example of answer:} \textit{The resulting code is:}

\begin{lstlisting}
class Counter(z,step) =
  cnt = z
  incr() = cnt <- step(cnt)


cnt1 = new Counter(0, + 1)
cnt2 = new Counter(1, * 1)

for i = 1 to 100 do
  cnt1.incr()
  cnt2.incr()
\end{lstlisting}

\ \\ 

\textbf{Points:} \textit{25\%.}

\ \\ 

\textbf{Grading:} \textit{All points for correct answer, correct function but wrong use or correct use but function with minor mistakes half points, wrong function and wrong use zero points.}

\ \\ 

\textbf{Associated learning goals:} \texttt{CLSABS}, \texttt{CLSDEF}.

\ \\ 


\paragraph{Question III: recursion on simple data structures} \ \\

\textbf{General shape of the question:} \textit{Define a recursive function that performs a simple operation on the nodes of a list or a tree. Show an instance of the tree, call its \texttt{Add} method, and indicate what the result would be.}

\ \\ 

\textbf{Concrete example of question:} \textit{Fill-in the body of the (recursive) \texttt{Add} method in order to add all elements of the binary tree.}

\begin{lstlisting}
struct Node
  Left
  Value
  Right
  
  ... constructor ...
  
  Add() = ?
\end{lstlisting}

\ \\ 

\textbf{Concrete example of answer:} \textit{The resulting code is:}

\begin{lstlisting}
struct Node
  Left
  Value
  Right
  
  ... constructor ...

  Add() = 
  	res = Value
  	if Left =/= null then
  	  res += Left.Add()
  	if Right =/= null then
  	  res += Right.Add()
  	return res
\end{lstlisting}

An instance of the tree would be:

\begin{lstlisting}
sample = new Node(new Node(10), 5, new Node(-10))
\end{lstlisting}

Calling

\begin{lstlisting}
sample.Add()
\end{lstlisting}

returns value \texttt{5}.

\ \\ 

\textbf{Points:} \textit{25\%.}

\ \\ 

\textbf{Grading:} \textit{All points for correct answer, correct function but wrong use or correct use but function with minor mistakes half points, wrong function and wrong use zero points.}

\ \\ 

\textbf{Associated learning goals:} \texttt{FUNREC}, \texttt{RECDATA}.

\ \\ 

\paragraph{Question IV: array processing} \ \\

\textbf{General shape of the question:} \textit{Define a loop that performs some simple operation on an array.}

\ \\ 

\textbf{Concrete example of question:} \textit{Define a loop that sums all positive elements of the array \texttt{numbers}. \texttt{numbers} contains only integers.}

\ \\ 

\textbf{Concrete example of answer:} \textit{The resulting code is:}

\begin{lstlisting}
sum = 0
for i = 0 to numbers.length - 1
  if numbers[i] > 0 then
    sum += numbers[i]
return sum
\end{lstlisting}

\ \\ 

\textbf{Points:} \textit{25\%.}

\ \\ 

\textbf{Grading:} \textit{All points for correct answer, otherwise zero points.}

\ \\ 

\textbf{Associated learning goals:} \texttt{ARR}.

\ \\ 

\subsection{Practical examination DEV II}
Every two lectures a practicum is due, for a total of \textbf{three practicums}. Each practicum must be handed in at the beginning of the second lecture, and that lecture is used to grade and discuss what was handed in. The overall procedure is made out of the following steps:

\begin{tabular}{| l | p{8cm} | p{4cm} |}
\hline
\textbf{Practicum} & \textbf{Activity} & \textbf{Hand-in} \\
\hline
1 & Write a series of functions that: 
\begin{inparaenum}[\itshape i\upshape)]
\item add the parameters together;
\item return the smallest of the parameters;
\item make the \textit{turtle} perform some set of movements; combine the movements with more functions;
\item recursively add numbers from \texttt{0} to \texttt{n}.
\end{inparaenum}. & Nothing \\
\hline
2 & Discuss and get grades. & Variables and expressions. \\
\hline
3 & Write a series of classes that:
\begin{inparaenum}[\itshape i\upshape)]
\item represent a fraction;
\item encapsulate a \textit{turtle} and make it perform some sets of movements;
\item represent a recursive list;
\item represent a recursive binary tree.
\end{inparaenum}. & Nothing \\
\hline
4 & Discuss and get grades. & \texttt{if} statements. \\
\hline
5 & Write a series of array-processing functions to:
\begin{inparaenum}[\itshape i\upshape)]
\item add all elements of an array;
\item add all even elements of an array;
\item find the smallest element of an array;
\item find a specific element of an array;
\item revert the order of the array elements.
\end{inparaenum}. & Nothing \\
\hline
6 & Discuss and get grades. & \texttt{for} and \texttt{while} statements. \\
\hline

\end{tabular}

\ \\

\end{document}
