\section{Course program}
	The course is structured into six lectures. The six lectures take place during the six weeks of the course, but are not necessarily in a one-to-one correspondance with the course weeks. For example, lectures one and two are fairly short and can take place during a single week.

	\subsection{Lecture 1 - data structures}
		The first lecture covers ...

		\paragraph*{Topics}
			\begin{itemize}
				\item Mechanism of abstraction;
				\item The necessity for data structures;
				\item Data structures in Python (class);
				\item Semantics (Heap, Stack);
				\item Layers of abstraction.
			\end{itemize}

		\paragraph*{Activities}
			\begin{itemize}
				\item Let students follow instructions;
				\item Introduce elements of state and let students follow instructions with state (\textit{take N/4 steps forward; N is your age});
				\item Introduce elements of writable state and let students follow instructions with writable state (\textit{take N/4 steps forward; N is written under the yellow sticker});
				\item Introduce elements of decision-making and let students follow instructions with state and decision making (\textit{if the sun is shining, then take N/4 steps forward; otherwise, go sit down});
				\item Introduce elements of iteration and let students follow instructions with state, decision making, and iteration (\textit{divide the students in teams, and let run some script battling for the toy farm}).
			\end{itemize}

			\subsection{Lecture 2 - lists}
				The second lecture covers ...

				\paragraph*{Topics}
					\begin{itemize}
						\item The need for a variable to contain an \textit{unknown} number of values;
						\item Abstraction of list: \texttt{Node (Head), Tail, Empty};
						\item Implementation of list (Python 3);
						\item Semantics of list: \texttt{Heap and Stack}
					\end{itemize}

				\paragraph*{Activities}
					\begin{itemize}
						\item ...
					\end{itemize}


			\subsection{Lecture 3 - functions}
				The third lecture covers ...

				\paragraph*{Topics}
					\begin{itemize}
						\item Abstraction operations (functions)
						\item The need for functions;
						\item Creating and using functions in Python;
						\item Formal and actual parameters and return;
						\item Brief introduction to: scope (local and global variables) and visibility;
						\item Syntax and semantics;
						\item Introduction to recursion;
					\end{itemize}

			\subsection{Lecture 4 - higher order functions and SQL}
				The fourth lecture covers ...

				\paragraph*{Topics}
					\begin{itemize}
						\item What are higher order functions (HOF's) and \textit{why we do need them}?
						\item Functions as parameter;
						\item Lambda: $\lambda$-expressions (syntax and semantics);
						\item Fundamental operations on list: \texttt{transform, filter, fold};
						\item Using HOF's;
						\item SQL vs list HOF's.
					\end{itemize}

				\paragraph*{Activities}
					Call upon students to solve small riddles related to sample Python scripts on:

					\begin{itemize}
						\item Integers, strings, floats, bools;
						\item Integer, string, float, and bool varialbes;
						\item Semantics and post-conditions on variable-assignments.
						\item Integers, strings, floats, bool expressions;
						\item Conditional expressions;
						\item Semantics and post-conditions on expressions and conditional expressions.
					\end{itemize}

			\subsection{Lecture 5 - methods}
				The fifth lecture covers ...

				\paragraph*{Topics}
					\begin{itemize}
						\item Joining functions (methods) and data to classes;
						\item Designing a class;
						\item Concrete implementation of a class;
						\item Syntax and semantics;
						\item special method names;
						\item rebuilding the list data structure;
						\item Brief introduction immutability and mutability.
					\end{itemize}

				\paragraph*{Activities}
					Call upon students to solve small riddles related to sample Python scripts on:

					\begin{itemize}
						\item \texttt{if-then} and \texttt{if-then-else} statements;
						\item how many possible final states of a program;
						\item semantics and post-conditions on conditional statements.
					\end{itemize}


			\subsection{Lecture 6 - collections library}
				The sixth (and last) lecture covers ...

				\paragraph*{Topics}
					\begin{itemize}
						\item repeated behaviors;
						\item \texttt{while} statements;
						\item (slightly informal) semantics;
						\item (more than) exponential explosion of potential control-paths;
						\item expressive power of \texttt{while};
						\item \texttt{for} statements;
						\item (slightly informal) semantics;
						\item \texttt{for} as a \textit{limited} form of \texttt{while}.
					\end{itemize}

				\paragraph*{Activities}
					Call upon students to solve small riddles related to sample Python scripts on:

					\begin{itemize}
						\item \texttt{while} and \texttt{for} loops;
						\item how many possible final states of a program;
						\item semantics and post-conditions on loops.
					\end{itemize}
