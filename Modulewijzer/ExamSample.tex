\section*{Exam sample}
What follows is a concrete example of the exam.


\paragraph{Question I: abstracting patterns with functions} \ \\

\textit{Define a \texttt{map} function to transform all elements of the input list (defined with \texttt{Empty} and \texttt{Node}, see Appendix) according to a given function \texttt{f}.}

\ \\ 

\textbf{Answer:} \textit{The resulting code is:}

\begin{lstlisting}
def map(l,f):
  if l.IsEmpty():
    return Empty()
  else:
    return Node(f(l.Head()), map(l.Tail(), f))
\end{lstlisting}

\ \\ 

\textbf{Points:} \textit{25\%.}

\ \\ 

\textbf{Grading:} \textit{All points for correct function, minor mistakes (wrong check, some elements might be missing, etc.) half points, wrong function (infinite recursion, iterative version, etc.) zero points.}

\ \\ 

\paragraph{Question II: runtime behaviour of functions} \ \\

\textit{Given the following function definition and a sample call, show stack and heap at all steps of the computation.}

\begin{lstlisting}
def filter(l, p):
    if not l.IsEmpty():
        if p(l.head):
            return Node(l.head, filter(l.tail, p))
        else:
            return filter(l.tail, p)
    else:
        return l
\end{lstlisting}

\ \\ 

\textbf{Answer:} \textit{Each call of the stack should contain a value for \texttt{l} and one for \texttt{p}. The heap should contain a value for each node of the list, and for the lambda function of \texttt{p}.}

\begin{lstlisting}
S: PC   filter   PC     l       p
    7    nil      2   ref(2)  ref(3)
    
H: 0             1                        2                      3
  [ ]   [head->2;tail->ref(0)]   [head->1;tail->ref(1)]   lambda x: x >= 2
\end{lstlisting}

\textit{After each step, the stack grows but the heap does not:}

\begin{lstlisting}
S: PC   filter   PC     l       p    filter   PC     l       p
    7    nil      5   ref(2)  ref(3)  nil      2   ref(1)  ref(3)
    
H: 0             1                        2                      3
  [ ]   [head->2;tail->ref(0)]   [head->1;tail->ref(1)]   lambda x: x >= 2
\end{lstlisting}

The rest follows similarly. Watch out for reconstruction of recursive result with correct elements wrt returned value of \texttt{p(l.head)}.

\textbf{Points:} \textit{25\%.}

\ \\ 

\textbf{Grading:} \textit{All points for all stack frames and values, half points for at least half correct stack frames and values, otherwise zero points.}

\ \\ 

\paragraph{Question III: classes} \ \\

\textbf{Concrete example of question:} \textit{Define a \texttt{Train} class with attributes:
\begin{itemize}
\item \texttt{Position} of the train in the map (a 2D vector, see Appendix)
\item amount of \texttt{Passengers}
\item amount of \texttt{Containers}
\end{itemize}
and methods:
\begin{itemize}
\item \texttt{TravelTo} that receives a \texttt{Station}\footnote{The \texttt{Station} class has at least the attributes \texttt{Position}, \texttt{WaitingPassengers}, \texttt{WaitingContainers}} as a destination and changes
\begin{itemize}
\item the \texttt{Position} of the train to the \texttt{Position} of the station
\item the \texttt{Passengers} of the train (and the \texttt{WaitingPassengers} of the station)
\item the \texttt{Containers} of the train (and the \texttt{WaitingContainers} of the station)
\end{itemize}
\end{itemize}}

\ \\ 

\textbf{Points:} \textit{25\%.}

\ \\ 

\textbf{Grading:} \textit{All points for all attributes and methods, half points for at least half correct attributes and methods, otherwise zero points.}

\ \\ 

\textbf{Answer:} \textit{The implemented class is:}

\begin{lstlisting}
class Station:
  def __init__(self, p, wp, wc):
    self.Position = p
    self.WaitingPassengers = wp
    self.WaitingContainers = wc

class Train:
  def __init__(self, p):
    self.Position = p
    self.Passengers = 0
    self.Containers = 0
  def NavigateTo(self, port):
    self.Position = port.Position
    self.Passengers = port.WaitingPassengers
    port.WaitingPassengers = 0
    self.Containers = port.WaitingContainers
    port.WaitingContainers = 0
\end{lstlisting}

\paragraph{Question IV: standard libraries} \ \\

\textbf{Concrete example of question:} \textit{Define a loop that adds all odd elements in a given Python list \texttt{l} which contains only integers. If the list is empty the result should be \texttt{0}. Finally, print the result.}

\ \\ 

\textbf{Points:} \textit{25\%.}

\ \\ 

\textbf{Grading:} \textit{All points for correct iteration and sum, half points for wrong use of indices or wrong iteration, zero points otherwise.}

\ \\ 

\textbf{Answer:} \textit{The implemented class is:}

\begin{lstlisting}
l = [1,2,3,4]
res = 0
for x in l:
  if x % 2 == 1:
    res = res + x
print(res)
\end{lstlisting}

\subsection{Exam appendix}

\paragraph{List implementation}
\begin{lstlisting}
class Empty:
  def IsEmpty(self): return True
Empty = Empty()

class Node:
  def IsEmpty(self): return False
  def Head(self): return self.Head
  def Tail(self): return self.Tail
  def __init__(self, x, xs):
    self.head = x
    self.tail = xs
\end{lstlisting}

\paragraph{Vector2 implementation}
\begin{lstlisting}
class Vector2:
  def __init__(self, x, y):
    self.X = x
    self.Y = y
  def Length(self):
    return math.sqrt(self.X * self.X + self.Y * self.Y)
  def __neg__(self):
    return Vector2(-self.X, -self.Y)
  def __add__(self, other):
    return Vector2(self.X + other.X, self.Y + other.Y)
  def __sub__(self, other):
    return self + (-other);
  def __mul__(self, k):
    return Vector2(self.X * k, self.Y * k)
  def __str__(self):
    return "(" + str(self.X) + "," + str(self.Y) + ")"
  def Zero(): 
    return Vector2(0.0, 0.0)
  def UnitX(): 
    return Vector2(1.0, 0.0)
  def UnitY(): 
    return Vector2(0.0, 1.0)
\end{lstlisting}
