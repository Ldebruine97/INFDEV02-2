\section*{Theoretical examination \modulecode}
The general shape of a theoretical exam for \texttt{DEV II} is made up of a series of highly structured open questions.


\paragraph{Question I: abstracting patterns with functions} \ \\

\textbf{General shape of the question:} \textit{Given a problem description, define one or more functions in order to solve the original problem.}

\ \\ 

\textbf{Concrete example of question:} \textit{Define a recursive \texttt{range} function to create a custom list (only use \texttt{Empty} and \texttt{Node}, see Appendix) with all the elements between two given numbers.}

\ \\ 

\textbf{Concrete example of answer:} \textit{The resulting code is:}

\begin{lstlisting}
def range(l, u):
  if l > u:
    return Empty()
  else:
    return Node(l, range(l+1,u))
\end{lstlisting}

\ \\ 

\textbf{Points:} \textit{25\%.}

\ \\ 

\textbf{Grading:} \textit{All points for correct function, minor mistakes (wrong check, some elements might be missing, etc.) half points, wrong function (infinite recursion, iterative version, etc.) zero points.}

\ \\ 

\textbf{Associated learning goals:} \texttt{FUNABS}, \texttt{FUNDEF}, \texttt{FUNREC}, \texttt{RECDATA}.

\ \\ 

\paragraph{Question II: runtime behaviour of functions} \ \\

\textbf{General shape of the question:} \textit{Given a function definition and a sample call, show stack and heap at all steps of the computation.}

\ \\ 

\textbf{Concrete example of question:} \textit{Given the following function definition and a sample call, show stack and heap at all steps of the computation.}

\begin{lstlisting}
def f(n):
  if n <= 1:
    return n
  else:
    return n * f(n-1)

f(3)
\end{lstlisting}

\ \\ 

\textbf{Concrete example of answer:} \textit{The last call of the stack is :}

\begin{lstlisting}
S: PC   f   PC   n   f   PC   n   f   PC   n
    7  nil   2   3  nil   2   2  nil   2   1
H: always empty
\end{lstlisting}

\textit{The stack will then unwind as follows:}

\begin{lstlisting}
S: PC   f   PC   n   f   PC   n   f   PC   n
    7  nil   2   3  nil   2   2   1   3   1
\end{lstlisting}

\ \\

\begin{lstlisting}
S: PC   f   PC   n   f   PC   n
    7  nil   2   3  2*1   4   2
\end{lstlisting}

\ \\

\begin{lstlisting}
S: PC   f   PC   n
    7  3*2   4   3
\end{lstlisting}

\ \\ 

\textbf{Points:} \textit{25\%.}

\ \\ 

\textbf{Grading:} \textit{All points for all stack frames and values, half points for at least half correct stack frames and values, otherwise zero points.}

\ \\ 

\textbf{Associated learning goals:} \texttt{FUNABS}, \texttt{FUNDEF}, \texttt{FUNREC}, \texttt{RECDATA}.

\ \\ 

\paragraph{Question III: classes} \ \\

\textbf{General shape of the question:} \textit{Given a description, give the implementation of a class and its methods in Python.}

\ \\ 

\textbf{Concrete example of question:} \textit{Define a \texttt{Counter} class with a single method, \texttt{Tick}, which increments the internal \texttt{cnt} of the class. Also provide an implementation of \texttt{\_\_str\_\_)}}

\ \\ 

\textbf{Concrete example of answer:} \textit{The resulting code is:}

\begin{lstlisting}
class Counter:
  def __init__(self):
    self.cnt = 0
  def Tick(self):
    self.cnt = self.cnt + 1
  def __str__(self):
    return "Ticked " + str(self.cnt) + " times"
\end{lstlisting}

\ \\ 

\textbf{Points:} \textit{25\%.}


\textbf{Grading:} \textit{All points for correct answer, half points for at least correct implementation of methods \_\_init\_\_ and Tick, otherwise zero points.}

\ \\ 

\textbf{Associated learning goals:} \texttt{CLSABS}, \texttt{CLSDEF}.

\ \\

\paragraph{Question IV: standard libraries} \ \\

\textbf{General shape of the question:} \textit{Define a loop that performs some simple operation on a standard data structure.}

\ \\ 

\textbf{Concrete example of question:} \textit{Define a loop that sums all positive elements of a Python list \texttt{l} which contains only integers. Finally, print the sum.}

\ \\ 

\textbf{Concrete example of answer:} \textit{The resulting code is:}

\begin{lstlisting}
sum = 0
for x in l:
  if x > 0:
    sum = sum + x
print(sum)
\end{lstlisting}

\ \\ 

\textbf{Points:} \textit{25\%.}

\ \\ 

\textbf{Grading:} \textit{All points for correct answer, otherwise zero points.}

\ \\ 

\textbf{Associated learning goals:} \texttt{ARR}.

\ \\
