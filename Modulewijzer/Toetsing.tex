\section{Assessment}
The course is tested with two exams: a series of practical assignments, a brief oral check of the practical assignments, and a theoretical exam. The final grade is determined as follows: \\

\texttt{if theoryGrade $\geq$ 75\% \& practicumCheckOK then return practicumGrade else return insufficient}

This means that the theoretical knowledge is a strict requirement in order to get the actual grade from the practicums, but it does not reflect your level of skill and as such does not further influence your grade.

\paragraph*{Motivation for grade}
A professional software developer is required to be able to program code which is, at the very least, \textit{correct}.

In order to produce correct code, we expect students to show:
\begin{inparaenum}[\itshape i\upshape)]
\item a foundation of knowledge about how a programming language actually works in connection with a simplified concrete model of a computer;
\item fluency when actually writing the code.
\end{inparaenum}

The quality of the programmer is ultimately determined by his actual code-writing skills, therefore the final grade comes only from the practicums. The quick oral check ensures that each student is able to show that his work is his own and that he has adequate understanding of its mechanisms. The theoretical exam tests that the required foundation of knowledge is also present to avoid away of programming that is exclusively based on intuition, but which is also grounded in concrete and precise knowledge about what each instruction does.


\subsection{Theoretical examination \modulecode}
The general shape of a theoretical exam for \texttt{\modulecode} is made up of a series of highly structured open questions. In each exam the content of the questions will change, but the structure of the questions will remain the same. For the structure (and an example) of the theoretical exam, see the appendix.


\subsection{Practical examination \modulecode}
Each week there is a mandatory assignment. The assignments of week 4, 5 and 6 will be graded. Each assignment is due the following week. The sum of the grades will be the $practicumGrade$. 
If the course is over and $practicumGrade$ is lower than $5,5$ then you can retry (herkansing) the practicum with one assignment which will test all learning goals and will replace the whole $practicumGrade$.
The following rules apply to the assignment:
\begin{itemize}
  \item The assignments are to be uploaded to N@tschool in the required space (Inlevermap);
  \item Only basic operations are allowed for the assignment unless explicit permitted otherwise; 
\end{itemize}
The oral check  (preferred during the practicums) is done on work uploaded to N@tschool:
\begin{itemize}
  \item two (2) questions per assignment about \textit{What does this line (these lines) do?}
  \item the exercise runs correctly
\end{itemize}


\subsection{Oral check \modulecode}
During the oral check, the teacher will verify ownership and competence with the code that was handed in during the practicum. This effectively determines the grade of the practicum. The procedure works as follows:

\begin{enumerate}
\item For each practicum, the teacher will \textbf{delete} some lines of code;
\item The student will then rewrite them from skratch;
\item Succesful restoring of the functionality will give the points for that assignment; failure in restoring the functionality will result in zero points for that practicum, independently of what was originally handed in.
\end{enumerate}
