\documentclass{beamer}
\usetheme[hideothersubsections]{HRTheme}
\usepackage{beamerthemeHRTheme}
\usepackage[utf8]{inputenc}
\usepackage{graphicx}
\usepackage[space]{grffile}

\title{Lists}

\author{TEAM INFDEV}

\institute{Hogeschool Rotterdam \\ 
Rotterdam, Netherlands}

\date{}

\begin{document}
\maketitle

\SlideSection{Introduction}
\SlideSubSection{Lecture topics}
\begin{slide}{
\item We now begin discussing specific, useful data structures
\item These are already well known and understood
\item Perfect for learning how a data structure is designed
\item We begin with lists
}\end{slide}

\SlideSection{Problem discussion}
\SlideSubSection{Introduction}
\begin{slide}{
\item So far we have been dealing with a single date in every variable
\item For example, integer \texttt{0} in variable \texttt{i}
\item Sometimes we need to store multiple things in the same variable
}\end{slide}

\SlideSubSection{Examples}
\begin{slide}{
\item All players
\item All the employees of the company
\item All the trucks on the road
\item All the aliens in the spaceship
\item All the alien spaceships in the fleet
\item ...
}\end{slide}

\begin{frame}[fragile]{With variables?}
\begin{codewithblock}{\pause \item Does this work? \item What if we have more or less than 10 trucks?}
truck1 = Truck(...)
truck2 = Truck(...)
...
truck10 = Truck(...)
\end{codewithblock}
\end{frame}

\SlideSection{General idea}
\SlideSubSection{Introduction}
\begin{slide}{
\item To solve this problem, we want to have all the data in a single variable
\item The variable contains thus an \textbf{unknown} number of values
\begin{itemize}
\item Might be empty
\item Might have only one element
\item Might have hundreds of elements
\item ...
\end{itemize}
}\end{slide}

\SlideSubSection{Description}
\begin{slide}{
\item To solve the issue, we will define an open-ended data structure
\item The list is built as a linear chain of \textbf{nodes}
\item In the simplest implementation, each node has
\begin{itemize}
\item a \textbf{value}
\item a reference to the \textbf{next} elements
\end{itemize}
\item We never really know how many elements we have in the list until we follow all the references through
\item A special case is the empty list, which has no element and no reference to the next elements
}\end{slide}

\begin{frame}[fragile]
\begin{codewithblock}{\item Consider a list with elements 3, 7, and 4 \item We need four nodes (the last is empty), all referencing the next}
+---+---+    +---+---+    +---+---+    +---+---+
| 3 | ------>| 7 | ------>| 4 | ------>|   |   |
+---+---+    +---+---+    +---+---+    +---+---+
\end{codewithblock}
\end{frame}

\begin{slide}{
\item A list of values is built as either of:
\begin{itemize}
\item An empty list \texttt{Empty}
\item A non-empty list containing the current value \texttt{v} and the rest of the list v\texttt{tail} \texttt{Node(v,tail)}
\end{itemize}
\item \textbf{A list with three integers would be?} \pause \texttt{Node(1,Node(2,Node(3,Empty)))}
\item \textbf{A list with two integers would be?} \pause \texttt{Node(1,Node(2,Empty))}
\item \textbf{An empty list would be?} \pause
\texttt{Empty}
}\end{slide}

\begin{slide}{
\item A list of values offers us three pieces of information:
\begin{itemize}
\item A boolean \texttt{IsEmpty} indicating whether or not the list is empty
\item The value \texttt{Value} of the current element of the list in case it is \textbf{not empty}
\item The rest \texttt{Tail} of the list in case it is \textbf{not empty}
\end{itemize}
\item Given a list \texttt{x}
\begin{itemize}
\item \textbf{We can check if it is empty with?} \pause \texttt{x.IsEmpty}
\item \textbf{We can read print its first value with?} \pause \texttt{x.Value}
\item \textbf{We can print its second value with?} \pause \texttt{x.Tail.Value}
\item \textbf{We can print its third value with?} \pause \texttt{x.Tail.Tail.Value}
\item ...
\end{itemize}
}\end{slide}

\SlideSection{Technical details}
\SlideSubSection{Introduction}
\begin{slide}{
\item How is this done in Python?
\item We shall build two data structures that, together, make up arbitrary lists
\item We begin with the blueprints
}\end{slide}

\begin{frame}[fragile]{The blueprint (\textbf{THIS IS NOT CODE!})}
\begin{codewithblock}{\pause \item How do we translate this to Python? \pause \item Each abstraction becomes a class \item Each field is assigned under \texttt{\_\_init\_\_} to \texttt{self}}
Abstraction Empty =
  IsEmpty, which is always true

Abstraction Node = 
  IsEmpty, which is always false
  Value, which contains the data of this element of the list
  Tail, which contains the remaining nodes of the list
\end{codewithblock}
\end{frame}

\begin{frame}[fragile]{The actual code}
\begin{lstlisting}
class Empty:
  def __init__(self):
      self.IsEmpty = True
Empty = Empty()

class Node:
  def __init__(self, value, tail):
      self.IsEmpty = False
      self.Value   = value
      self.Tail    = tail
\end{lstlisting}

\textbf{Note: we are switching to Python 3!}
\end{frame}

\SlideSubSection{Examples of list usage}
\begin{slide}{
\item We now wish to build a list with our data structures
\item We will build a list based on the input of the user
\item User specifies how many, and which elements must go in the list
}\end{slide}

\begin{frame}[fragile]{Examples of list usage}
\begin{memorytable}
{|c|}
{PC}
{1}
{|c|}
{}
{}
\end{memorytable} \ \\

\begin{lstlisting}
l = Empty
count = int(input("How many elements?"))
for i in range(0, count):
  v = int(input("Insert the next element")
  l = Node(v, l)
\end{lstlisting}

\pause 

\begin{memorytable}
{|c|c|c|c|c|}
{PC & l & count & i & v}
{\red{5} & \red{ref(0)} & \red{5} & \red{0} & \red{80085} }
{|c|}
{0}
{ [ IsEmpty $\mapsto$ True ] }
\end{memorytable}
\end{frame}

\begin{frame}[fragile]{Examples of list usage}
\begin{memorytable}
{|c|c|c|c|c|}
{PC & l & count & i & v}
{5 & ref(0) & 5 & 0 & 80085 }
{|c|}
{0}
{ [ IsEmpty $\mapsto$ True ] }
\end{memorytable} \ \\

\begin{lstlisting}
l = Empty
count = int(input("How many elements?"))
for i in range(0, count):
  v = int(input("Insert the next element")
  l = Node(v, l)
\end{lstlisting}

\pause 

\begin{memorytable}
{|c|c|c|c|c|}
{PC & l & count & i & v }
{\red{3} & ref(1) & 5 & 0 & 80085 }
{|c|c|}
{0 & 1}
{ [ IsEmpty $\mapsto$ True ] & [ IsEmpty $\mapsto$ False; Value $\mapsto$ 80085; Tail $\mapsto$ ref(0)] }
\end{memorytable}
\end{frame}

\begin{frame}[fragile]{Examples of list usage}
\begin{memorytable}
{|c|c|c|c|c|}
{PC & l & count & i & v }
{5 & ref(1) & 5 & 1 & 8078 }
{|c|c|}
{0 & 1}
{ ... & [ IsEmpty $\mapsto$ False; Value $\mapsto$ 80085; Tail $\mapsto$ ref(0)] }
\end{memorytable}
 \ \\

\begin{lstlisting}
l = Empty
count = int(input("How many elements?"))
for i in range(0, count):
  v = int(input("Insert the next element")
  l = Node(v, l)
\end{lstlisting}

\pause 

\begin{memorytable}
{|c|c|c|c|c|}
{PC & l & count & i & v }
{5 & ref(1) & 5 & 1 & 8078 }
{|c|c|c|}
{0 & 1 & 2}
{ ... & ... & [ IsEmpty $\mapsto$ False; Value $\mapsto$ 8078; Tail $\mapsto$ ref(1)] }
\end{memorytable}
\end{frame}

\begin{slide}{
\item We now wish to use the list we just built
\item Specifically, we will print all its elements
\item \textbf{How many elements does it have?}
\pause
\item Unknown: it is specified by the user!
}\end{slide}

\begin{frame}[fragile]{Examples of list usage}
\begin{memorytable}
{|c|c|}
{PC & l }
{1 & ref(2)}
{|c|c|c|}
{0 & 1 & 2}
{ [ E $\mapsto$ T ] & [ E $\mapsto$ F; V $\mapsto$ 2; T $\mapsto$ ref(0)] & [ E $\mapsto$ F; V $\mapsto$ 3; T $\mapsto$ ref(1)] }
\end{memorytable}
 \ \\

\begin{lstlisting}
x = l
while not(x.IsEmpty):
  print(x.Value)
  x = x.Tail
\end{lstlisting}

\pause 

\begin{memorytable}
{|c|c|c|}
{PC & l & x}
{2 & ref(2) & \red{ref(2)}}
{|c|c|c|}
{0 & 1 & 2}
{ [ E $\mapsto$ T ] & [ E $\mapsto$ F; V $\mapsto$ 2; T $\mapsto$ ref(0)] & [ E $\mapsto$ F; V $\mapsto$ 3; T $\mapsto$ ref(1)] }
\end{memorytable}
\end{frame}

\begin{frame}[fragile]{Examples of list usage}
\begin{memorytable}
{|c|c|c|}
{PC & l & x}
{2 & ref(2) & ref(2)}
{|c|c|c|}
{0 & 1 & 2}
{ [ E $\mapsto$ T ] & [ E $\mapsto$ F; V $\mapsto$ 2; T $\mapsto$ ref(0)] & [ E $\mapsto$ F; V $\mapsto$ 3; T $\mapsto$ ref(1)] }
\end{memorytable}
 \ \\

\begin{lstlisting}
x = l
while not(x.IsEmpty):
  print(x.Value)
  x = x.Tail
\end{lstlisting}

What gets printed? \pause \texttt{H[S[x]][Value] = H[2][Value] = 3}

\pause

\begin{memorytable}
{|c|c|c|}
{PC & l & x}
{\red{3} & ref(2) & \red{ref(2)}}
{|c|c|c|}
{0 & 1 & 2}
{ [ E $\mapsto$ T ] & [ E $\mapsto$ F; V $\mapsto$ 2; T $\mapsto$ ref(0)] & [ E $\mapsto$ F; V $\mapsto$ 3; T $\mapsto$ ref(1)] }
\end{memorytable}
\end{frame}

\begin{frame}[fragile]{Examples of list usage}
\begin{memorytable}
{|c|c|c|}
{PC & l & x}
{3 & ref(2) & ref(2)}
{|c|c|c|}
{0 & 1 & 2}
{ [ E $\mapsto$ T ] & [ E $\mapsto$ F; V $\mapsto$ 2; T $\mapsto$ ref(0)] & [ E $\mapsto$ F; V $\mapsto$ 3; T $\mapsto$ ref(1)] }
\end{memorytable}
 \ \\

\begin{lstlisting}
x = l
while not(x.IsEmpty):
  print(x.Value)
  x = x.Tail
\end{lstlisting}

\textbf{Where is x.Tail}? \pause \texttt{H[S[x]][Tail] = H[2][Tail] = ref(1)}

\pause

\begin{memorytable}
{|c|c|c|}
{PC & l & x}
{\red{4} & ref(2) & \red{ref(1)}}
{|c|c|c|}
{0 & 1 & 2}
{ [ E $\mapsto$ T ] & [ E $\mapsto$ F; V $\mapsto$ 2; T $\mapsto$ ref(0)] & [ E $\mapsto$ F; V $\mapsto$ 3; T $\mapsto$ ref(1)] }
\end{memorytable}
\end{frame}

\begin{frame}[fragile]{Examples of list usage}
\begin{memorytable}
{|c|c|c|}
{PC & l & x}
{3 & ref(2) & ref(1)}
{|c|c|c|}
{0 & 1 & 2}
{ [ E $\mapsto$ T ] & [ E $\mapsto$ F; V $\mapsto$ 2; T $\mapsto$ ref(0)] & [ E $\mapsto$ F; V $\mapsto$ 3; T $\mapsto$ ref(1)] }
\end{memorytable}
 \ \\

\begin{lstlisting}
x = l
while not(x.IsEmpty):
  print(x.Value)
  x = x.Tail
\end{lstlisting}

\textbf{Where is x.Tail}? \pause \texttt{H[S[x]][Tail] = H[1][Tail] = ref(0)}

\pause

\begin{memorytable}
{|c|c|c|}
{PC & l & x}
{\red{4} & ref(2) & \red{ref(0)}}
{|c|c|c|}
{0 & 1 & 2}
{ [ E $\mapsto$ T ] & [ E $\mapsto$ F; V $\mapsto$ 2; T $\mapsto$ ref(0)] & [ E $\mapsto$ F; V $\mapsto$ 3; T $\mapsto$ ref(1)] }
\end{memorytable}
\end{frame}

\begin{frame}[fragile]{Examples of list usage}
\begin{memorytable}
{|c|c|c|}
{PC & l & x}
{2 & ref(2) & ref(0)}
{|c|c|c|}
{0 & 1 & 2}
{ [ E $\mapsto$ T ] & [ E $\mapsto$ F; V $\mapsto$ 2; T $\mapsto$ ref(0)] & [ E $\mapsto$ F; V $\mapsto$ 3; T $\mapsto$ ref(1)] }
\end{memorytable}
 \ \\

\begin{lstlisting}
x = l
while not(x.IsEmpty):
  print(x.Value)
  x = x.Tail
\end{lstlisting}

\textbf{What is the value of x.IsEmpty}? \pause \texttt{H[S[x]][IsEmpty] = H[0][IsEmpty] = True}

\pause

\begin{memorytable}
{|c|c|c|}
{PC & l & x}
{\red{5} & ref(2) & ref(0)}
{|c|c|c|}
{0 & 1 & 2}
{ [ E $\mapsto$ T ] & [ E $\mapsto$ F; V $\mapsto$ 2; T $\mapsto$ ref(0)] & [ E $\mapsto$ F; V $\mapsto$ 3; T $\mapsto$ ref(1)] }
\end{memorytable}
\end{frame}

\SlideSection{In-class homework}
\SlideSubSection{Implement the following (on paper)}
\begin{slide}{
\item Read a list from the user input
\item Remove all odd numbers
\item A ``volunteer'' runs the steps on paper with the memory model
}\end{slide}

\begin{slide}{
\item Read a list from the user input
\item Sum all its values
\item A ``volunteer'' runs the steps on paper with the memory model
}\end{slide}

\begin{slide}{
\item Read a list from the user input
\item Reverse it
\item A ``volunteer'' runs the steps on paper with the memory model
}\end{slide}

\begin{slide}{
\item Read two lists from the user input
\item Append the second to the first (concatenate them)
\item A ``volunteer'' runs the steps on paper with the memory model
}\end{slide}

\SlideSection{Conclusion}
\SlideSubSection{Lecture topics}
\begin{slide}{
\item What we solved today was the issue of representing multiple data inside a single variable
\item We used a simple data structure, the \textbf{list}
\item We showed how we can consume (use) the list through looping
}\end{slide}

\begin{thankyou}{
}\end{thankyou}

\end{document}

\begin{slide}{
\item ...
}\end{slide}

\begin{frame}[fragile]
\begin{lstlisting}
...
\end{lstlisting}
\end{frame}
