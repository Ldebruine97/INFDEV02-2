\documentclass{beamer}
\usetheme[hideothersubsections]{HRTheme}
\usepackage{beamerthemeHRTheme}
\usepackage[utf8]{inputenc}
\usepackage{graphicx}
\usepackage[space]{grffile}

\title{Classes as data structures with methods}

\author{TEAM INFDEV}

\institute{Hogeschool Rotterdam \\ 
Rotterdam, Netherlands}

\date{}

\begin{document}
\maketitle

\SlideSection{Introduction}
\SlideSubSection{Lecture topics}
\begin{slide}{
\item In this lecture we will ``close the circle'' of data structures and functions
\item We will show how it is possible to form a new, even more powerful abstraction
\item We will define \textbf{classes} as the joining of functions and data structures
}\end{slide}

\SlideSection{Problem discussion}
\SlideSubSection{Introduction}
\begin{slide}{
\item Functions on lists always take a list as input
\item This list is understood as the main \textbf{subject} of the computation
\item We would like to create a stronger visual and semantic link between the subject of a computation and the computation itself
}\end{slide}

\SlideSubSection{Examples}
\begin{slide}{
\item \texttt{length} of a list
\item \texttt{sum} of a list
\item \texttt{find} in a list a given element
\item \texttt{map} of a list wrt some transformation
\item ...
}\end{slide}

\SlideSection{General idea}
\SlideSubSection{Introduction}
\begin{slide}{
\item To solve the problem, we will create \texttt{classes}
\item A \texttt{class} is the union of a data structures and its characterizing functions
\item The functions inside a class are known as \texttt{methods}
}\end{slide}

\SlideSubSection{Blueprint of a class}
\begin{slide}{
\item What are the fundamental attributes?
\item What properties do the attributes have (relationships, etc.)?
\item What are the fundamental methods of the class?
\item What properties do the attributes have (relationships, returned values, etc.)?
}\end{slide}

\begin{frame}[fragile]{The blueprint of the list class (\textbf{THIS IS NOT CODE!})}
\begin{lstlisting}
Abstraction List =
  the list may be Empty, with
    no attributes
    IsEmpty() method returns True
    Length method returns 0
    Map(f) method returns the empty list
    Filter(p) method returns the empty list
    ...
    
  the list may be a Node
    head attribute (the value of the element of this node)
    tail attribute (the rest of the list)
    IsEmpty() method returns False
    Length method returns the length of the list
    Map(f) method returns the the transformed list wrt function f
    Filter(p) method returns the list with only elements respecting p
    ...
\end{lstlisting}
\end{frame}

\SlideSubSection{Design of a class}
\begin{slide}{
\item The hardest part of building a class is its design
\item How do we build a reasonable class?
\item What is a bad implementation?
}\end{slide}

\begin{slide}{
\item The same logic that applies to the design of functions applies to classes
\item \textbf{Encapsulation}\footnote{Also called \textbf{information hiding}} is the central property of both
\item A class is well encapsulated if 
\begin{itemize}
\item it offers a clear interaction surface by not exposing its internals in a dangerous way
\item it is a cohesive unit that offers a single, clearly defined set of related services
\end{itemize}
}\end{slide}

\begin{slide}{
\item Thanks to \textbf{encapsulation}, a program can be built as a series of independent units
\item These units are \textbf{loosely coupled}, in the sense that a change in the implementation (but not the methods offered) of one unit does not break the others\footnote{Think about a faster implementation, for example.}
}\end{slide}

\begin{slide}{
\item A bad example would be a \texttt{ListOrPlayer} class which contains a list or a player, with methods and fields such as:
\begin{itemize}
\item \texttt{IsList}, \texttt{IsPlayer}
\item \texttt{Name}, \texttt{Score}, \texttt{Weapon}, ...
\item \texttt{Length}, \texttt{Map}, \texttt{Filter}, ...
\end{itemize}
\item Why is it a bad example?
\pause
\item Because it is not a clear unit, but two at the same time
}\end{slide}

\begin{slide}{
\item A bad example would be a \texttt{List} which \textit{leaks} parts of implementation
\item A ``leaky'' list would have strange methods that rely on specific external usage patterns like:
\begin{itemize}
\item \texttt{ComputeFirstPartOfLength} that computes the length of the first half of the list
\item \texttt{ComputeSecondPartOfLength} that computes the length of the second half of the list
\item \texttt{GetLengthParts} that returns the lengths of the two half lists
\end{itemize}
\item Why is it a bad example?
\pause
\item Because the implementation of length only happens if the user of the list calls it properly
}\end{slide}

\SlideSection{Technical details}
\SlideSubSection{Introduction}
\begin{slide}{
\item We have already seen how we can define a class in Python, just without methods (beside \texttt{\_\_init\_\_})
\item Adding methods is surprisingly simple
\item Simply declare a function within the class body, with the usual syntax
\begin{itemize}
\item The function must\footnote{There are exceptions, we discuss them later.} have a special parameter \texttt{self}, which is the first parameter
\end{itemize}
}\end{slide}

\begin{frame}[fragile]{Concrete class implementation}
\begin{codewithblock}{\pause \item Notice the \texttt{self} parameter of each method}
class Empty:
  def IsEmpty(self): 
    return True
  def Length(self):
    return 0
\end{codewithblock}
\end{frame}

\begin{frame}[fragile]{Concrete class implementation}
\begin{codewithblock}{\pause \item Notice the \texttt{self} parameter of each method}
class Node:
  def IsEmpty(self): return False
  def __init__(self, x, xs):
    self.head = x
    self.tail = xs
  def Length(self):
    return 1 + self.tail.Length()
\end{codewithblock}
\end{frame}

\SlideSubSection{Using classes}
\begin{slide}{
\item Usage of classes remains quite intuitive
\item Instance a class by using its name and giving parameters that will end up in \texttt{\_\_init\_\_}
\item The returned class can be assigned to a variable
\item The methods and attributes of the class can be called with a \texttt{.} between the instance of a class and the method/attribute name
}\end{slide}

\begin{frame}[fragile]{Actual class usage}
\begin{lstlisting}
l = Node(1, Empty())
print(l.Length())
\end{lstlisting}
\end{frame}

\SlideSubSection{Constructor}
\begin{slide}{
\item You may now have realized that \texttt{\_\_init\_\_} is just another method\footnote{A special one, in that it is called automatically by Python, but a regular method nonetheless}, the \textbf{constructor} of a class
\item It is called transparently when creating an instance of a class
\item We call it with a parameter less (\texttt{self}), which is given automatically by Python as a new empty container
\item \texttt{Node(1,Empty())} invokes the constructor with two parameters, \texttt{self} is implicit
}\end{slide}

\SlideSubSection{Methods}
\begin{slide}{
\item Methods work like the constructor
\item We call them with a parameter less (\texttt{self}), which is given automatically by Python as the object from which the method was called
\item \texttt{l.Length()} calls method \texttt{Length} with zero parameters, \texttt{self} is implicitly passed as \texttt{l}
}\end{slide}

\SlideSubSection{Syntax and semantics of methods}
\begin{slide}{
\item The only new element of semantics is indeed the passing of \texttt{self}
\item This is described with a code transformation that happens at runtime
\item Method call is not really a new feature, but rather a handy way to use existing features
}\end{slide}

\begin{slide}{
\item Whenever Python encounters a call such as \texttt{x.M(p1, ..., p2} then it \textit{transforms} it into a similar call \texttt{C.M(x, p1, ..., pn)} where \texttt{C} is the type of \texttt{x} in memory
\item This way \texttt{x} automatically becomes \texttt{self} inside \texttt{M}

$$\langle x.M(p_1,\dots,p_n), S, H \rangle \rightarrow \langle H[S[x]][C].M(x,p_1,\dots,p_n) \rangle$$

\item This presupposes that the heap contains, for each class, an additional attribute \texttt{C} that is the class of declaration

\begin{tabular}{|c|c|c|}
\hline
... & n & ... \\
\hline
... & [...; C $\mapsto$ TypeOfInstance] & ... \\
\hline
\end{tabular}
}\end{slide}

\begin{slide}{
\item Methods defined with a \texttt{self} parameter are called \textbf{instance methods}
\item Using them is bound to the calling context (represented by \texttt{self})
\item Some methods can be defined without a \texttt{self} parameter
\item Such methods are called \textbf{static methods}, because they are independent of a specific instance
}\end{slide}

\begin{slide}{
\item Static methods are used by specifying the name of the class to which the method belongs to \texttt{C.M(p1,...,pn)}
\item There is nothing special about them, they just behave like any function call
\item They are useful to group functionality related to a single class that is not related to the class instances
}\end{slide}

\SlideSubSection{Special method names}
\begin{slide}{
\item Some special method names are reserved by Python
\item These names represent operators that are automatically called by Python
\item A typical example is \texttt{\_\_str\_\_}, which is automatically invoked whenever \texttt{print} or \texttt{str} are used
}\end{slide}

\begin{slide}{
\item Other special names represent operators
\item For example: 
\begin{itemize}
\item \texttt{\_\_le\_\_} is called whenever \texttt{x <= y} was used
\item \texttt{\_\_add\_\_} is called whenever \texttt{x + y} was used
\item \texttt{\_\_lshift\_\_} is called whenever \texttt{x << y} was used
\item ...
\end{itemize}
}\end{slide}

\SlideSubSection{Examples}
\begin{slide}{
\item Let's begin with a simple example
\item A counter that keeps track of how many times some event has happened
\item The constructor of the counter starts the count at zero
\item A \texttt{Tick} method increments the count
\item A pretty printer to nicely format instances when they are output
}\end{slide}

\begin{frame}[fragile]{``Counter'' example}
\begin{lstlisting}
class Counter:
  def __init__(self):
    self.cnt = 0
  def Tick(self, n):
    self.cnt = self.cnt + n
  def __str__(self):
    return "Ticked " + str(self.cnt) + " times."

c = Counter()
for i in range(0, 20):
  c.Tick(i)
  print(c)
\end{lstlisting}

\textbf{What does this program do?}

\pause

\textbf{Prints the ``ticked'' message for} \texttt{0, 1, 3, 6, 10}
\end{frame}


\begin{slide}{
\item Let's move to an almost full implementation of lists
\item Lists come in two flavours: \texttt{Empty}, and \texttt{Node}
\item We want at least the following methods:
\begin{itemize}
\item \texttt{IsEmpty}
\item String conversion
\item \texttt{<<} to create lists more easily
\item \texttt{Sum} to add all elements of the list
\item \texttt{Length} to find the list length
\item \texttt{Map} to transform all elements of the list
\item \texttt{Filter} to remove some elements of the list
\item \texttt{Fold} to collapse the list
\end{itemize}
}\end{slide}

\begin{frame}[fragile]{Let's start with the empty list}
\begin{lstlisting}
class Empty:
  def IsEmpty(self): return True
  def __str__(self):
    return "[]"
  def __rlshift__(self, v):
    return Node(v, self)
  def Sum(self):
    return 0
  def Length(self):
    return 0
  def Map(self,f):
    return self
  def Filter(self,p):
    return self
  def Fold(self,f,z):
    return z
Empty = Empty()
\end{lstlisting}
\end{frame}

\begin{frame}[fragile]{The non-empty list}
\begin{lstlisting}
class Node:
  def IsEmpty(self): return False
  def Head(self): return self.Head
  def Tail(self): return self.Tail
  def __init__(self, x, xs):
    self.head = x
    self.tail = xs
  def __rlshift__(self, v):
    return Node(v, self)
  def __str__(self):
    return str(self.head) + "<<" + str(self.tail)
  def Sum(self):
    return self.head + self.tail.Sum()
  def Length(self):
    return 1 + self.tail.Length()
  def Map(self,f):
    return Node(f(self.head), self.tail.Map(f))
  def Filter(self,p):
    xs = self.tail.Filter(p)
    if p(self.head):
      return Node(self.head, xs)
    else:
      return xs
  def Fold(self,f,z):
    return f(self.head, self.tail.Fold(f,z))
\end{lstlisting}
\end{frame}

\SlideSubSection{Assignments}
\begin{slide}{
\item Find the implementation (or the slides) online and open them to reference \texttt{Node} and \texttt{Empty}
\item What does \texttt{l = 1 << (2 << (3 << (4 << Empty)))} do?
\begin{itemize}
\pause 
\item It creates list \texttt{Node(1, Node(2, Node(3, Node(4, Empty))))}
\end{itemize}
\item Why did we write \texttt{Empty} instead of \texttt{Empty()}?
\begin{itemize}
\pause 
\item Aesthetics: we define a single instance of \texttt{Empty = Empty()} to avoid recalling the constructor every time
\end{itemize}
}\end{slide}

\SlideSubSection{What does this program print?}
\begin{frame}[fragile]{Using lists}
\begin{codewithblock}{\pause \item \texttt{'4', '2<<3<<4<<5', '2<<4', '10', '24']}}
l = 1 << (2 << (3 << (4 << Empty)))
print("length(" + str(l) + ") = " + str(l.Length()))
print("incr(" + str(l) + ") = " + str(l.Map(lambda x: x + 1)))
print("even(" + str(l) + ") = " + str(l.Filter(lambda x: x % 2 == 0)))
print("sum(" + str(l) + ") = " + str(l.Sum()))
print("mul(" + str(l) + ") = " + str(l.Fold(lambda x,y: x * y, 1)))
\end{codewithblock}
\end{frame}

\SlideSubSection{Let's see build a 2D vector}
\begin{slide}{
\item We want things to move over the screen
\item We define 2D vectors to store the position of objects
\item 2D vectors contain only \texttt{X}, \texttt{Y} attributes
\item Methods are:
\begin{itemize}
\item Vector addition, subtraction, negation, multiplication by scalar
\item Length of the vector (according to Pythagoras' theorem)
\item Some static methods to get some special vectors (null, (0,1), (1,0)
\end{itemize}
}\end{slide}

\begin{frame}[fragile]{Implementation of vector 2D}
\begin{lstlisting}
class Vector2:
  def __init__(self, x, y):
    self.X = x
    self.Y = y
  def Length(self):
    return math.sqrt(self.X * self.X + self.Y * self.Y)
  def __neg__(self):
    return Vector2(-self.X, -self.Y)
  def __add__(self, other):
    return Vector2(self.X + other.X, self.Y + other.Y)
  def __sub__(self, other):
    return self + (-other);
  def __mul__(self, k):
    return Vector2(self.X * k, self.Y * k)
  def __str__(self):
    return "(" + str(self.X) + "," + str(self.Y) + ")"
  def Zero(): 
    return Vector2(0.0, 0.0)
  def UnitX(): 
    return Vector2(1.0, 0.0)
  def UnitY(): 
    return Vector2(0.0, 1.0)
\end{lstlisting}
\end{frame}

\SlideSubSection{What does this program print?}
\begin{frame}[fragile]{Let's build a 2D vector}
\begin{codewithblock}{\pause \item \texttt{(10.0, -2.0)]}}
v = Vector2.UnitX() * 10.0 - Vector2.UnitY() * 2.0
print(str(v))
\end{codewithblock}
\end{frame}

\SlideSubSection{Let's make a car}
\begin{slide}{
\item A car has a \texttt{Position} and a \texttt{Velocity}, both \texttt{Vector2}'s
\item A car also has \texttt{Gas} in the tank
\item A car can \texttt{Travel}: \texttt{Position} moves forward, \texttt{Gas} is burned
}\end{slide}

\begin{frame}[fragile]{Implementation of car}
\begin{lstlisting}
class Car:
  def __init__(self, p, v, g):
    self.Position = p
    self.Velocity = v
    self.Gas = g
  def Travel(self, dt):
    if self.Gas > 0.0:
      self.Position = self.Position + self.Velocity * dt
      self.Gas = self.Gas - 1.0 * dt
  def __str__(self):
    return "A car at " + str(self.Position) + " with a tank of " + str(self.Gas) + " liters"
\end{lstlisting}
\end{frame}

\SlideSubSection{What does this program print?}
\begin{frame}[fragile]{Let's build a car!}
\begin{codewithblock}{\pause \item \texttt{A car at (2.0,0.0) with a tank of 8.0 liters} \item \texttt{A car at (4.0,0.0) with a tank of 6.0 liters} \item \texttt{A car at (6.0,0.0) with a tank of 4.0 liters} \item \texttt{A car at (8.0,0.0) with a tank of 2.0 liters} \item \texttt{A car at (10.0,0.0) with a tank of 0.0 liters}}
c = Car(Vector2.Zero(), Vector2.UnitX(), 10.0)
while(c.Gas > 0.0):
  c.Travel(2.0)
  print(c)
\end{codewithblock}
\end{frame}

\SlideSection{A few advanced design topics}
\SlideSubSection{General idea}
\begin{slide}{
\item There are countless ways to design classes
\item Each of these has advantages and disadvantages
\item It is a design discipline, and as such ``artsier'' than strictly needed
\item Experience will show the way
\item For the moment, we just illustrate one such possibility
}\end{slide}

\SlideSubSection{Immutable and mutable classes}
\begin{slide}{
\item Not all classes modify the values of their attributes as methods are called
\item Some classes can be designed to be \textbf{immutable}
\item An instance never changes values after creation
\item Methods \textbf{return a new instance with the new attribute values} instead of changing the attribute values of the starting instance
}\end{slide}

\begin{slide}{
\item Never changing attributes is a powerful technique
\item Sharing of instances is safer, because unexpected changes will never happen
\item \textbf{Can you think of examples when this might be useful?}
\pause
\item \textbf{Breaking another class which assumes that the given instance will always remain as expected}
\item \textbf{Multiple threads}
\item \textbf{Rollback/checkpoints/transactional systems}
\item ...
}\end{slide}

\SlideSubSection{Let's make an immutable car}
\begin{slide}{
\item A car has a \texttt{Position} and a \texttt{Velocity}, both \texttt{Vector2}'s
\item A car also has \texttt{Gas} in the tank
\item A car can \texttt{Travel}: \textbf{a new car is returned where} \texttt{Position} is moved forward, \texttt{Gas} is burned
}\end{slide}

\begin{frame}[fragile]{Implementation of an immutable car}
\begin{lstlisting}
class ImmutableCar:
  def __init__(self, p, v, g):
    self.Position = p
    self.Velocity = v
    self.Gas = g
  def Travel(self, dt):
    if self.Gas > 0.0:
      return Car(self.Position + self.Velocity * dt, self.Velocity, self.Gas - 1.0 * dt)
    else:
      return self
  def __str__(self):
    return "A car at " + str(self.Position) + " with a tank of " + str(self.Gas) + " liters"
\end{lstlisting}
\end{frame}

\SlideSubSection{What does this program print?}
\begin{frame}[fragile]{Let's build an immutable car!}
\begin{codewithblock}{\pause \item \texttt{A car at (2.0,0.0) with a tank of 8.0 liters} \item \texttt{A car at (4.0,0.0) with a tank of 6.0 liters} \item \texttt{A car at (6.0,0.0) with a tank of 4.0 liters} \item \texttt{A car at (8.0,0.0) with a tank of 2.0 liters} \item \texttt{A car at (10.0,0.0) with a tank of 0.0 liters}}
c = ImmutableCar(Vector2.Zero(), Vector2.UnitX(), 10.0)
while(c.Gas > 0.0):
  c = c.Travel(2.0)
  print(c)
\end{codewithblock}
\end{frame}

\SlideSubSection{Rolling back time}
\begin{slide}{
\item Suppose we stored the initial value of the car \texttt{c0 = ImmutableCar(Vector2.Zero(), Vector2.UnitX(), 10.0)} before the loop
\item Then we just run the loop
\item \textbf{Can we ``roll back time'' by going back to the initial value of the car?}
\pause 
\item Yes!
\item \textbf{Was this possible with the first implementation?}
\pause 
\item No!
}\end{slide}

\begin{frame}[fragile]{Rolling back time}
\begin{lstlisting}
c = ImmutableCar(Vector2.Zero(), Vector2.UnitX(), 10.0)
c0 = c
while(c.Gas > 0.0):
  c = c.Travel(2.0)
  print(c)
print(c0)
\end{lstlisting}
\textbf{What is stored in} \texttt{c0}\textbf{?}
\pause
\textbf{The initial value of the car, unchanged!}
\end{frame}

\SlideSubSection{Choices, choices}
\begin{slide}{
\item Why would you use one design instead of the other?
\item Mutable classes have 
\begin{itemize}
\item the advantage that they are simple to build and intuitive
\item the disadvantage that they are ``destructive'', in the sense that all modification destroys previous information that might still be in use
\end{itemize}
\item Immutable classes have 
\begin{itemize}
\item the advantage that they are not destructive
\item the disadvantage that they are more complex to build and structure
\end{itemize}
\item Mutable classes for current state of an object
\item Immutable classes for things that are not supposed to change (vectors, numbers) or that might need to be rolled back
}\end{slide}

\SlideSubSection{Lecture topics}
\begin{slide}{
\item What problem did we solve today, and how?
}\end{slide}

\SlideSection{Conclusion}
\SlideSubSection{Lecture topics}
\begin{slide}{
\item In this lecture we have ``closed the circle'' of data structures and functions
\item We have defined \textbf{classes} as the joining of functions and data structures
\item We have seen a series of class implementation examples in action
\item We have discussed different design balances that might come into consideration when choosing the structure of a class
}\end{slide}

\begin{thankyou}
\end{thankyou}

\end{document}

\begin{slide}{
\item ...
}\end{slide}

\begin{frame}[fragile]
\begin{lstlisting}
...
\end{lstlisting}
\end{frame}
