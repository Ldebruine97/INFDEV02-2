\documentclass{beamer}
\usetheme[hideothersubsections]{HRTheme}
\usepackage{beamerthemeHRTheme}
\usepackage[utf8]{inputenc}
\usepackage{graphicx}
\usepackage[space]{grffile}

\title{Standard collections library}

\author{TEAM INFDEV}

\institute{Hogeschool Rotterdam \\ 
Rotterdam, Netherlands}

\date{}

\begin{document}
\maketitle

\SlideSection{Introduction}
\SlideSubSection{Lecture topics}
\begin{slide}{
\item Building one's collections is often not needed
\item Most modern languages couple a standard library with the runtime
\item In this lecture we shall list the various built-in collections of Python ($\geq 3$)
}\end{slide}

\SlideSection{Problem discussion}
\SlideSubSection{Introduction}
\begin{slide}{
\item How do we represent collections of data in Python?
\item What sort of collections are available?
\item What properties does each collection offer?
}\end{slide}

\begin{slide}{
\item Collections are not always straightforward \textbf{bags} of data
\item Collections should be thought of as \textbf{bags with structure}
}\end{slide}

\SlideSubSection{Bags?}
\begin{slide}{
\item A bag of data contains multiple values
\item The size of the bag is usually variable
\item Elements may be accessed via some dynamic mechanism
\begin{itemize}
\item For example, \textit{find the third element}
\end{itemize}
}\end{slide}

\SlideSubSection{Bags \textbf{with structure}?}
\begin{slide}{
\item Structure of data refers to how data is stored
\item There are \textbf{a lot} of possible structures of data
\begin{itemize}
\item Lack of duplicates
\item Connections between values
\item Order-preservation
\item ...
\end{itemize}
\item We cannot really cover them all, we shall just brush the surface and remind you of the existance of \url{www.google.com}
}\end{slide}

\SlideSubSection{Example structures we shall encounter}
\begin{textslide}{
\begin{description}
\item[Tuple] Fixed number of elements in a fixed order
\item[List] Dynamic number of elements in a fixed order
\item[Sets] Dynamic number of unique elements
\item[Maps] Dynamic number of unique keys mapped to non-unique values
\end{description}
}\end{textslide}

\SlideSubSection{Example tuples - \textbf{(THIS IS NOT CODE)}}
\begin{slide}{
\item A fixed number of values, kept in a fixed order
\item \texttt{(0,5)}
\item \texttt{(1,``Hello!'')}
\item \texttt{(``Hello!'', ``World!'', 100)}
}\end{slide}

\SlideSubSection{Example lists - \textbf{(THIS IS NOT CODE)}}
\begin{slide}{
\item A dynamic number of values, kept in a fixed order
\item \texttt{[]}
\item \texttt{[0; 5; 10; 20; 100; 20]}
\item \texttt{[``Hello!''; ``World!''; 100]}
}\end{slide}

\SlideSubSection{Example sets - \textbf{(THIS IS NOT CODE)}}
\begin{slide}{
\item A dynamic number of values, without specific order, and no duplicates
\item \texttt{\{\}}
\item \texttt{\{0; 5; 10; 20; 100\} = \{0; 5; 10; 20; 100; 20\}}
\item \texttt{\{``Hello!''; ``World!''; 100\}}
}\end{slide}

\SlideSubSection{Example maps - \textbf{(THIS IS NOT CODE)}}
\begin{slide}{
\item A dynamic number of keys, each connected to its value\footnote{The stack and the heap are maps!}
\item \texttt{[]}
\item \texttt{[0 $\mapsto$ "John"; 1 $\mapsto$ "Jack"; $\mapsto$ "Jill"]}
}\end{slide}

\SlideSection{General idea}
\SlideSubSection{Introduction}
\begin{slide}{
\item Python offers a specific data structure for each of these containers
\item These data structures are fast to write due to specialized syntax
\item These data structures are fast to use due to internal optimizations
\pause
\item These data structures are \textbf{not the only option}, sometimes you might need to build your own
}\end{slide}

\SlideSubSection{Description}
\begin{slide}{
\item We will need to learn new syntax and new behaviours
\item Keep in mind that this new syntax is not as essential as the basic syntax of Python
\item This new batch of syntax is just aesthetics
\pause
\item We could remove it and the language would not lose expressive power
}\end{slide}

\SlideSection{Tuples}
\begin{frame}[fragile]{Tuples}
Making an empty tuple
\begin{lstlisting}
empty = ()
\end{lstlisting}
\end{frame}

\begin{frame}[fragile]{Tuples}
Making a tuple with just one element\footnote{\textbf{NOTE the trailing comma!!!}}
\begin{lstlisting}
singleton = 'hello',
\end{lstlisting}
\end{frame}

\begin{frame}[fragile]{Tuples}
Making a tuple with more than one element
\begin{lstlisting}
t = (12345, 54321, 'hello!')
\end{lstlisting}
\end{frame}

\begin{frame}[fragile]{Tuples}
Breaking it down
\begin{lstlisting}
t = (12345, 54321, 'hello!')
t_x = t[0]
x, y, z = t
\end{lstlisting}
\end{frame}

\begin{frame}[fragile]{Tuples}
Nesting tuples
\begin{lstlisting}
t = (12345, 54321, 'hello!')
u = t, (1, 2, 3, 4, 5)
\end{lstlisting}
\end{frame}

\begin{frame}[fragile]{Tuples}
Pretty printing
\begin{lstlisting}
t = (12345, 54321, 'hello!')
print(t)
\end{lstlisting}
\end{frame}

\SlideSection{Lists}
\begin{frame}[fragile]{Lists}
Making an empty list
\begin{lstlisting}
l = []
\end{lstlisting}
\end{frame}

\begin{frame}[fragile]{Lists}
Making a non-empty list
\begin{lstlisting}
a = [-1, 1, 66.25, 333, 333, 1234.5]
\end{lstlisting}
\end{frame}

\begin{frame}[fragile]{Lists}
Finding elements
\begin{lstlisting}
a[0]
\end{lstlisting}
\end{frame}

\begin{frame}[fragile]{Lists}
Finding ranges
\begin{lstlisting}
a[1:3]
\end{lstlisting}
\end{frame}

\begin{frame}[fragile]{Lists}
Removing elements
\begin{lstlisting}
del a[0]
\end{lstlisting}
\end{frame}

\begin{frame}[fragile]{Lists}
Removing ranges
\begin{lstlisting}
del a[1:3]
\end{lstlisting}
\end{frame}

\begin{frame}[fragile]{Lists}
Removing whole range
\begin{lstlisting}
del a[:]
\end{lstlisting}
\end{frame}

\begin{frame}[fragile]{Lists}
Transforming a whole list (\texttt{map})\footnote{Notice the \texttt{list} function that finalizes creation of the list.}
\begin{lstlisting}
squares = list(map(lambda x: x**2, range(10)))
\end{lstlisting}
\end{frame}

\begin{frame}[fragile]{Lists}
List comprehensions \textbf{(THIS IS NOT ACTUAL CODE!)}
\begin{codewithblock}{\item For each value \texttt{i1} in list \texttt{l1} \item ... \item For each value \texttt{iN} in list \texttt{lN} \item If \texttt{COND} is true \item Add \texttt{element} to resulting list}
[<<element>> for <<i1>> in <<l1>> ... for <<iN>> in <<lN>> ... if <<COND>>]
\end{codewithblock}
\end{frame}

\begin{frame}[fragile]{Lists}
Building lists with list comprehensions
\begin{lstlisting}
squares = [x**2 for x in range(10)]
\end{lstlisting}
\end{frame}

\begin{frame}[fragile]{Lists}
Building lists with list comprehensions, multiple iterators, and conditionals
\begin{lstlisting}
[(x, y) for x in [1,2,3] for y in [3,1,4] if x != y]
\end{lstlisting}
\end{frame}

\begin{frame}[fragile]{Lists}
And a lot of other things...\footnote{Just be aware of their existence, learn them as you need!}
\begin{lstlisting}
list.append(x) #Add an item to the end of the list
list.extend(L) #Extend the list by appending all the items in L
list.insert(i, x) #Insert an item at a given position
list.remove(x) #Remove the first item x from the list
list.pop([i]) #Remove the item at the given position in the list
list.clear() #Remove all items from the list
list.index(x) #Return the index in the list of the first x
list.count(x) #Return the number of times x appears in the list
list.sort(key=None, reverse=False) #Sort the items of the list in place
list.reverse() #Reverse the elements of the list in place
list.copy() #Return a shallow copy of the list
\end{lstlisting}
\end{frame}

\SlideSection{Sets}
\begin{frame}[fragile]{Sets}
Making an empty set\footnote{Watch out: \textbf{it is not} \texttt{\{\}}}
\begin{lstlisting}
p = set()
\end{lstlisting}
\end{frame}

\begin{frame}[fragile]{Sets}
Making a non-empty set
\begin{lstlisting}
basket = {'apple', 'orange', 'apple', 'pear', 'orange', 'banana'}
\end{lstlisting}
\end{frame}

\begin{frame}[fragile]{Sets}
Lookup in a set
\begin{lstlisting}
hasOranges = 'orange' in basket
\end{lstlisting}
\end{frame}

\begin{frame}[fragile]{Sets}
Union and intersection
\begin{lstlisting}
inAorInB = a | b
inAandInB = a & b
\end{lstlisting}
\end{frame}

\begin{frame}[fragile]{Sets}
Difference
\begin{lstlisting}
inAbutNotB = a - b
inAorBButNotBoth = a ^ b
\end{lstlisting}
\end{frame}

\begin{frame}[fragile]{Sets}
Set comprehensions
\begin{lstlisting}
{x for x in 'abracadabra' if x not in 'abc'}
\end{lstlisting}
\end{frame}

\SlideSection{Maps}
\begin{frame}[fragile]{Sets}
Making an empty map
\begin{lstlisting}

\end{lstlisting}
\end{frame}

\begin{frame}[fragile]{Sets}
Maps
\begin{lstlisting}
tel = {'jack': 4098, 'sape': 4139}
print(tel)
tel['guido'] = 4127
print(tel)
print(tel['jack'])
del tel['sape']
print(tel)
tel['irv'] = 4127
print(tel)
print(list(tel.keys()))
print(sorted(tel.keys()))
print('guido' in tel)
print('jack' not in tel)
print(dict([('sape', 4139), ('guido', 4127), ('jack', 4098)]))
knights = {'gallahad': 'the pure', 'robin': 'the brave'}
for k, v in knights.items():
  print(k, v)
\end{lstlisting}
\end{frame}


\SlideSection{Iterators and generators}


\SlideSection{Conclusion}
\SlideSubSection{Lecture topics}
\begin{slide}{
\item What problem did we solve today, and how?
}\end{slide}

\begin{frame}{This is it!}
\center
\fontsize{18pt}{7.2}\selectfont
The best of luck, and thanks for the attention!
\end{frame}

\end{document}

\begin{slide}{
\item ...
}\end{slide}

\begin{frame}[fragile]
\begin{lstlisting}
...
\end{lstlisting}
\end{frame}
