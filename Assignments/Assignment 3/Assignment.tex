\documentclass[10pt,a4paper]{article}
\usepackage[latin1]{inputenc}
\usepackage{enumitem}
\usepackage{amsfonts}
\usepackage{amssymb}
\usepackage{hyperref}
\usepackage{listings}
\usepackage{xcolor}
\usepackage{amsmath}
\usepackage{wasysym}
\usepackage{graphicx}
\begin{document}
\title{Assignment 3\\DEV 2 - Year 2015-2016\\\textit{HOF's}}
\author{The Dev TEAM}
\maketitle


\section{Goal and description}
The goal is to improve your design and implementation skills on high order functions. For this assignment you have to reuse your code form the Assignment 2.
\begin{itemize}
\item design and implement the boat data structure  
\item move through the scene both cars and boats
\end{itemize}
	
	
\section{Software requirements}
To work with the simulation you need \texttt{PyGame 3.4} and \texttt{Python 3.4}. You can download the \texttt{PyGame 3.4 x86} \href{https://bitbucket.org/pygame/pygame/downloads/pygame-1.9.2a0-hg_ea3b3bb8714a.win32-py3.4.msi}{\textcolor{red}{click me}}. PyGame is a set of Python modules designed for writing games.	The simulation comes with a \textit{template project}. The template is available on N@school and GitHub under the voice \texttt{Assignment 1}.
reuse the car data structure 
extend the car with can remove and texture
\section{Details}

\paragraph{HOF's - revision}
In the following we show the high order functions needed to solve this assignment: \texttt{map} and \texttt{filter}. 

\paragraph{map} is a high order function that takes as parameters a list \texttt{l} and a transformation function \texttt{f} and returns a \textit{new} list made of all elements of \texttt{l} transformed by \texttt{f}. In order to use the \texttt{map} function the transformation function \texttt{f} \textbf{must} always the transformed value. In the following example elements of a list of numbers are all increased by one.
\begin{lstlisting}[frame=single]
numbers = Node(1, Node(2, Node(3, Empty)))
new_numbers = map(numbers, lambda n: n + 1)
\end{lstlisting}

\paragraph{filter} is a high order function that takes as parameters a list \texttt{l} and a predicate function \texttt{p} and returns a new list made of only the elements of \texttt{l} that satisfy the predicate \texttt{p}. In order to use the \texttt{filter} function the predicate \texttt{p} must always return a boolean value. In the following example from a list of number we select the even numbers.
\begin{lstlisting}[frame=single]
numbers = Node(1, Node(2, Node(3, Empty)))
new_numbers = map(numbers, lambda n: n % 2 == 0)
\end{lstlisting}

\noindent
\textbf{N.B.,} You need to study the high order functions and methods for this assignment.
	
\section{Tasks}	

\paragraph{Task 1} \textbf{[HOF} adapt the assignment 2 code so to include HOF's in your code. Precisely, your HOF's should replace your previous \textit{draw all cars/draw all boats} and \textit{update all boats/update all cars}. Design and implement such code.


\noindent
\textbf{Hint} For drawing your boats and cars you might need an other HOF called \texttt{iterate}. The \texttt{iterate} function iterates the elements of a list and applies a generic function \texttt{f} to all elements of the list. 


\noindent
\textbf{N.B.,} The function \texttt{iterate} should not change the list dimension or the state of its values. Design such \texttt{iterate} function and use it accordingly.

\section{Submission and deadline}

\noindent
Contribution: \textit{Groups of 2 students is allowed with individual responsibility}

\noindent
What: \textit{One PDF per student for all code + comments (comments: explain your code)}

\noindent
When: \textit{The Friday of week 7}

\noindent
Where: \textit{On N@school}


\huge
\centering
GOOD LUCK!!! The Dev TEAM \smiley

	
\end{document}