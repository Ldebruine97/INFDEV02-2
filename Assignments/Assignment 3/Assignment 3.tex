\documentclass[10pt,a4paper]{article}
\usepackage[latin1]{inputenc}
\usepackage{enumitem}
\usepackage{amsfonts}
\usepackage{amssymb}
\usepackage{hyperref}
\usepackage{listings}
\usepackage{xcolor}
\usepackage{amsmath}
\usepackage{wasysym}
\usepackage{graphicx}
\begin{document}
\title{Assignment 3\\DEV 2 - Year 2015-2016\\\textit{HOF's}}
\author{The Dev TEAM}
\maketitle


\section{Goal and description}
The goal is to improve your design and implementation skills about \textit{HOF's} (high order functions). For this assignment you may your code form Assignment 2.

	
\section{Software requirements}
To work with the simulation you need \texttt{PyGame 3.4} and \texttt{Python 3.4}. You can download the \texttt{PyGame 3.4 x86} from \href{https://bitbucket.org/pygame/pygame/downloads/pygame-1.9.2a0-hg_ea3b3bb8714a.win32-py3.4.msi}{\textcolor{red}{https://bitbucket.org/pygame/pygame/ downloads/pygame-1.9.2a0-hg\_ea3b3bb8714a.win32-py3.4.msi}}. PyGame is a set of Python modules designed for writing games.	The simulation comes with a \textit{template project}. The template is available on N@school and GitHub under the voice \texttt{Assignment 3}.

\section{Details}
\paragraph{HOF's - revision}
In the following we show the HOF's necessary to solve this assignment: \texttt{map} and \texttt{filter}. 

\paragraph{map} is a high order function that takes as parameters a list \texttt{l} and a transformation function \texttt{f} and returns a \textit{new} list made of all elements of \texttt{l} transformed by \texttt{f}. In order to use the \texttt{map} function the transformation function \texttt{f} \textbf{must} always return the transformed value. In the following example elements of a list of numbers are all increased by one:
\begin{lstlisting}[frame=single,basicstyle=\ttfamily]
numbers = Node(1, Node(2, Node(3, Empty)))
updated_numbers = map(numbers, lambda n: n + 1)
\end{lstlisting}

\paragraph{filter} is a high order function that takes as parameters a list \texttt{l} and a predicate function \texttt{p} (which evaluates a boolean condition) and returns a new list made of only the elements of \texttt{l} that satisfy the predicate \texttt{p} (for which the condition return \texttt{True}). In order to use the \texttt{filter} function the predicate \texttt{p} must always return a boolean value. In the following example we select the even numbers from a list of numbers:
\begin{lstlisting}[frame=single,basicstyle=\ttfamily]
numbers = Node(1, Node(2, Node(3, Empty)))
filtered_numbers = filter(numbers, lambda n: n % 2 == 0)
\end{lstlisting}

\paragraph{iterate} is a high order function that takes as parameters a list \texttt{l} and a predicate \texttt{p} and applies the predicate to each element of the list \texttt{l}. Note that the function returns nothing, which means that we just ``do'' something ``with'' each element, ideally leaving the original list intact. In the following example we print every element of a list:
\begin{lstlisting}[frame=single,basicstyle=\ttfamily]
numbers = Node(1, Node(2, Node(3, Empty)))
iterate(numbers, lambda n: print(n))
\end{lstlisting}
	
\section{Tasks}	
\paragraph{Task 1} \textbf{[HOF]} adapt your code from the Assignment 2 so to include HOF's. Precisely, your HOF's should replace your previous \textit{draw all cars/draw all boats} and \textit{update all boats/update all cars}. Design and implement such code.

\vspace{0.3cm}
\noindent
\textbf{Hint 1} For drawing your boats and cars use \texttt{iterate}. \\
\noindent
\textbf{Hint 2} The function \texttt{iterate} is not provided. Try to design and implement it.\\
\noindent
\textbf{Hint 3} For moving boats and cars use \texttt{map}.\\
\noindent
\textbf{Hint 4} For removing boats and cars that reached their destination use \texttt{filter}.\\


\section{Submission and deadline}

\noindent
Contribution: \textit{Groups of 2 students is allowed with individual responsibility}

\noindent
What: \textit{One PDF per student for all code + comments (comments: explain your code)}

\noindent
When: \textit{The Friday of week 7}

\noindent
Where: \textit{On N@school}


\huge
\centering
GOOD LUCK!!! The Dev TEAM \smiley

	
\end{document}