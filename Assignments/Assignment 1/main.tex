\documentclass[10pt,a4paper]{article}
\usepackage[latin1]{inputenc}
\usepackage{enumitem}
\usepackage{amsfonts}
\usepackage{amssymb}
\usepackage{hyperref}

\usepackage{amsmath}
\usepackage{wasysym}
\usepackage{graphicx}
\begin{document}
\title{Assignment 1\\DEV 2 - Year 2015-2016}
\author{The Dev TEAM}
\maketitle


\section{Goal and description}
The goal is to improve your design and implementation skills on data structures. For this purpose we created an \textit{incomplete} simulation of a city, where only the city and it's roads are rendered. \textbf{Your task is to design and implement cars that are able to move randomly through the city}. 
	
	
\section{Software requirements}
To work with the simulation you need \texttt{PyGame 3.4} and \texttt{Python 3.4}. You can download the \texttt{PyGame 3.4 x86} \href{https://bitbucket.org/pygame/pygame/downloads/pygame-1.9.2a0-hg_ea3b3bb8714a.win32-py3.4.msi}{here}. PyGame is a set of Python modules designed for writing games.	The simulation comes with a \textit{template project}. The template is available on N@school and GitHub under the voice \texttt{Assignment 1}.

\section{Details}

\paragraph{Classes}
As you will see in the template we have implemented some classes for you: a \texttt{Node} and a \texttt{Tile} data structure, accordingly available in \texttt{Node.py} and \texttt{Tile.py}. We recommend you to read them carefully and to understand their attributes.

The class \texttt{Tile} has a \texttt{Properties} attribute. Elements in \texttt{Properties} gives you information about the current node. For example a property could be \texttt{NotTraversable}, which means that this node is not traversable; or \texttt{Parking}, which means that this node is a parking place; etc. You can make you own properties if necessary.


\noindent
NB. You need to study those structures and codes before you start with your implementation. 
	
\paragraph{Game.py}

We also we provide you a main loop in \texttt{Game.py}. The \texttt{Main} function is the entry point of the game. Precisely the in the \texttt{Main} you find the a block of code which runs indefinitely the game. Within the block of code we call the functions \texttt{Update} and \texttt{Draw} to update the scene logic and display the scene elements accordingly.

\noindent
Inside \texttt{Game.py} search for the function \texttt{Update}. Update takes as parameters the lists of cars to update and returns a new collection of cars (note a car might get filtered in case it enters a parking tile)

\section{Tasks}	

\paragraph{Task 1} \textit{Design} and the \texttt{Car} data structure that should at least provide the following attributes:
\begin{itemize}[noitemsep,nolistsep]
	\item A position, which references the node the car is in
\end{itemize}

\paragraph{Task 2} In \texttt{Update} \textit{implement} the behavior of your cars.
\begin{itemize}[noitemsep,nolistsep]
	\item \textbf{Move} your cars randomly through the city (based on the current node of the car) and avoid non traversable nodes
	\item \textbf{Add} new cars after a condition is met. For example add a new car every 5 seconds (check \texttt{speed}).
	\item \textbf{Remove} a car from \texttt{car\_list} if it enters a parking place
\end{itemize}

\paragraph{Task 3} Draw all cars. We provide you a function \texttt{Draw} that takes as input a list of cars. Use the hint we provide you inside the \texttt{Draw} function.

\section{Submission and deadline}

\noindent
Contribution: \textit{Groups of 2 students is allowed with individual responsibility}

\noindent
What: \textit{One PDF per student for all code + comments (comments: explain your code)}

\noindent
When: \textit{The Friday of week 6}

\noindent
Where: \textit{On N@school}


\vspace{5cm}
\huge
\centering
GOOD LUCK!!! The Dev TEAM \smiley 

	
\end{document}